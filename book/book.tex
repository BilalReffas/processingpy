\documentclass[10pt,german,]{book}
\usepackage{lmodern}
\usepackage{amssymb,amsmath}
\usepackage{ifxetex,ifluatex}
\usepackage{fixltx2e} % provides \textsubscript
\ifnum 0\ifxetex 1\fi\ifluatex 1\fi=0 % if pdftex
  \usepackage[T1]{fontenc}
  \usepackage[utf8]{inputenc}
\else % if luatex or xelatex
  \ifxetex
    \usepackage{mathspec}
    \usepackage{xltxtra,xunicode}
  \else
    \usepackage{fontspec}
  \fi
  \defaultfontfeatures{Mapping=tex-text,Scale=MatchLowercase}
  \newcommand{\euro}{€}
\fi
% use upquote if available, for straight quotes in verbatim environments
\IfFileExists{upquote.sty}{\usepackage{upquote}}{}
% use microtype if available
\IfFileExists{microtype.sty}{%
\usepackage{microtype}
\UseMicrotypeSet[protrusion]{basicmath} % disable protrusion for tt fonts
}{}
\usepackage[a5paper,left=3cm,right=2cm,bottom=2cm,top=2cm]{geometry}
\ifxetex
  \usepackage[setpagesize=false, % page size defined by xetex
              unicode=false, % unicode breaks when used with xetex
              xetex]{hyperref}
\else
  \usepackage[unicode=true]{hyperref}
\fi
\hypersetup{breaklinks=true,
            bookmarks=true,
            pdfauthor={},
            pdftitle={},
            colorlinks=true,
            citecolor=blue,
            urlcolor=blue,
            linkcolor=magenta,
            pdfborder={0 0 0}}
\urlstyle{same}  % don't use monospace font for urls
\setlength{\parindent}{0pt}
\setlength{\parskip}{6pt plus 2pt minus 1pt}
\setlength{\emergencystretch}{3em}  % prevent overfull lines
\setcounter{secnumdepth}{0}
\ifxetex
  \usepackage{polyglossia}
  \setmainlanguage{german}
\else
  \usepackage[german]{babel}
\fi

\date{}
\usepackage{fancyhdr}
\pagestyle{fancy}
\fancyhead{}
\fancyfoot{}

\fancyhead[EL]{\thepage}
\fancyhead[OR]{\thepage}

\fancyhead[ER]{\nouppercase{\rightmark}}
\fancyhead[OL]{\nouppercase{\leftmark}}

\begin{document}

{
\hypersetup{linkcolor=black}
\setcounter{tocdepth}{2}
\tableofcontents
}
\chapter{Einleitung}\label{einleitung}

\chapter{Download und Installation}\label{download-und-installation}

\chapter{Start: Rotkäppchen und die drei
Tanten}\label{start-rotkuxe4ppchen-und-die-drei-tanten}

\chapter{Punkte und Pixel}\label{punkte-und-pixel}

\section{Turmite}\label{turmite}

\section{Wir backen uns ein
Mandelbrötchen}\label{wir-backen-uns-ein-mandelbruxf6tchen}

\section{Pixel-Array versus set()}\label{pixel-array-versus-set}

\section{Julia-Menge}\label{julia-menge}

\section{Schnelle Bildmanipulation: Das
Pixel-Array}\label{schnelle-bildmanipulation-das-pixel-array}

\subsection{Fantastic Feather Fractal}\label{fantastic-feather-fractal}

\chapter{Linien}\label{linien}

\section{Anschauliche Mathematik: Die
Schmetterlingskurve}\label{anschauliche-mathematik-die-schmetterlingskurve}

\section{Der Lorenz-Attraktor, eine Ikone der
Chaos-Theorie}\label{der-lorenz-attraktor-eine-ikone-der-chaos-theorie}

\chapter{Shapes}\label{shapes}

\section{For Your Eyes Only -- Processing.py zieht
Kreise}\label{for-your-eyes-only-processing.py-zieht-kreise}

\section{Spaß mit Kreisen: Konfetti}\label{spauxdf-mit-kreisen-konfetti}

\section{Syntaktischer Zucker: »with« in
Processing.py}\label{syntaktischer-zucker-with-in-processing.py}

\section{Spaß mit Kreisen (2) in Processing.py: Cantor-Käse und
mehr}\label{spauxdf-mit-kreisen-2-in-processing.py-cantor-kuxe4se-und-mehr}

\section{Weitere geometrische
Grundformen}\label{weitere-geometrische-grundformen}

\section{Eine analoge Uhr aus
Kreisbögen}\label{eine-analoge-uhr-aus-kreisbuxf6gen}

\section{Visualisierung: Die
Sonntagsfrage}\label{visualisierung-die-sonntagsfrage}

\section{Der Baum des Pythagoras}\label{der-baum-des-pythagoras}

\chapter{Text(verarbeitung) in
Processing.py}\label{textverarbeitung-in-processing.py}

\section{Spaß mit Processing.py:
Rentenuhr}\label{spauxdf-mit-processing.py-rentenuhr}

\chapter{Bildverarbeitung mit
Processing.py}\label{bildverarbeitung-mit-processing.py}

\section{Jeder sein kleiner Warhol}\label{jeder-sein-kleiner-warhol}

\section{Filter für die
Bildverarbeitung}\label{filter-fuxfcr-die-bildverarbeitung}

\section{Pointillismus}\label{pointillismus}

\section{Noch mehr Pointillismus}\label{noch-mehr-pointillismus}

\chapter{Animationen}\label{animationen}

\section{Ein kleiner roter
Luftballon}\label{ein-kleiner-roter-luftballon}

\chapter{Spaß mit (SVG-) Shapes: Pinguine im
Eismeer}\label{spauxdf-mit-svg--shapes-pinguine-im-eismeer}

\chapter{Objkete und Klassen mit
Kitty}\label{objkete-und-klassen-mit-kitty}

\section{Hallo Hörnchen -- Hallo Kitty
revisited}\label{hallo-huxf6rnchen-hallo-kitty-revisited}

\section{Moving Kitty}\label{moving-kitty}

\section{Klasse Kitty!}\label{klasse-kitty}

\section{»Cute Planet« mit
Processing.py}\label{cute-planet-mit-processing.py}

\section{Fluffy Planet}\label{fluffy-planet}

\chapter{Zelluläre Automaten}\label{zelluluxe4re-automaten}

\section{Das Demokratie-Spiel}\label{das-demokratie-spiel}

\section{Frösche und Schildkröten oder: Wie entsteht
Segregation?}\label{fruxf6sche-und-schildkruxf6ten-oder-wie-entsteht-segregation}

\section{Der Waldbrand-Simulator}\label{der-waldbrand-simulator}

\chapter{3D mit Processing.py}\label{d-mit-processing.py}

\section{Kugeln und Kisten}\label{kugeln-und-kisten}

\section{Und es geht doch: Kugeln und
Texturen}\label{und-es-geht-doch-kugeln-und-texturen}

\section{Die Erde ist eine Kiste}\label{die-erde-ist-eine-kiste}

\section{Licht und Schatten}\label{licht-und-schatten}

\section{Einen Globus basteln}\label{einen-globus-basteln}

\chapter{Einen eigenen Wetterbericht mit
OpenWeatherMap}\label{einen-eigenen-wetterbericht-mit-openweathermap}

\chapter{WordCram: Processing.py und eine Processing (Java)
Bibliothek}\label{wordcram-processing.py-und-eine-processing-java-bibliothek}

\chapter{Running Orc mit
Processing.py}\label{running-orc-mit-processing.py}

\section{Running Orc in vier
Richtungen}\label{running-orc-in-vier-richtungen}

\section{Ork mit Kollisionserkennung}\label{ork-mit-kollisionserkennung}

\section{Ein Ork im Labyrinth}\label{ein-ork-im-labyrinth}

\section{Der autonome Ork}\label{der-autonome-ork}

\section{Drei Orks und ein Held}\label{drei-orks-und-ein-held}

\chapter{Exkurs Rauhnächte: Spaß mit
Processing.py}\label{exkurs-rauhnuxe4chte-spauxdf-mit-processing.py}

\chapter{Exkurs: Walking Pingus}\label{exkurs-walking-pingus}

\chapter{Das Avoider Game}\label{das-avoider-game}

\section{Game Stage 1}\label{game-stage-1}

\section{Stage 2}\label{stage-2}

\section{Stage 3: Sternenhimmel}\label{stage-3-sternenhimmel}

\section{Stage 4: PowerUp und
PowerDown}\label{stage-4-powerup-und-powerdown}

\section{Nachtrag: Avoider Game Stage
4a}\label{nachtrag-avoider-game-stage-4a}

\chapter{Epilog}\label{epilog}

\chapter{Anhang}\label{anhang}

\section{Literaturverzeichnis}\label{literaturverzeichnis}

\section{Index}\label{index}

\end{document}
